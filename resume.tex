%%%%%%%%%%%%%%%%%
% This is an sample CV template created using altacv.cls
% (v1.1.2, 1 February 2017) written by LianTze Lim (liantze@gmail.com). Now compiles with pdfLaTeX, XeLaTeX and LuaLaTeX.
%
%% It may be distributed and/or modified under the
%% conditions of the LaTeX Project Public License, either version 1.3
%% of this license or (at your option) any later version.
%% The latest version of this license is in
%%    http://www.latex-project.org/lppl.txt
%% and version 1.3 or later is part of all distributions of LaTeX
%% version 2003/12/01 or later.
%%%%%%%%%%%%%%%%

%% If you need to pass whatever options to xcolor
\PassOptionsToPackage{dvipsnames}{xcolor}

%% If you are using \orcid or academicons
%% icons, make sure you have the academicons
%% option here, and compile with XeLaTeX
%% or LuaLaTeX.
% \documentclass[10pt,a4paper,academicons]{altacv}

%% Use the "normalphoto" option if you want a normal photo instead of cropped to a circle
% \documentclass[10pt,a4paper,normalphoto]{altacv}

\documentclass[10pt,a4paper]{altacv}

%% AltaCV uses the fontawesome and academicon fonts
%% and packages.
%% See texdoc.net/pkg/fontawecome and http://texdoc.net/pkg/academicons for full list of symbols.
%%
%% Compile with LuaLaTeX for best results. If you
%% want to use XeLaTeX, you may need to install
%% Academicons.ttf in your operating system's font
%% folder.


% Change the page layout if you need to
\geometry{left=1cm,right=9cm,marginparwidth=7.25cm,marginparsep=0.75cm,top=0.5cm,bottom=1cm}

% Change the font if you want to.

% If using pdflatex:
\usepackage[utf8]{inputenc}
\usepackage[T1]{fontenc}
\usepackage[default]{lato}
\usepackage[none]{hyphenat}
\usepackage[document]{ragged2e}

% If using xelatex or lualatex:
% \setmainfont{Lato}

% Change the colours if you want to
\definecolor{AccentGreen}{HTML}{339966}
\definecolor{AcePurple}{HTML}{800080}
\definecolor{DarkerPurple}{HTML}{5b005b}
\definecolor{SlateGrey}{HTML}{2E2E2E}
\definecolor{LightGrey}{HTML}{666666}
\definecolor{SomeBlue}{HTML}{4286f4}
\colorlet{heading}{AcePurple}%Sepia}
\colorlet{accent}{DarkerPurple}%AccentGreen}
\colorlet{emphasis}{SlateGrey}
\colorlet{body}{LightGrey}

% Change the bullets for itemize and rating marker
% for \cvskill if you want to
\renewcommand{\itemmarker}{{\small\textbullet}}
\renewcommand{\ratingmarker}{\faCircle}

\begin{document}
\name{Galen Guyer}
\tagline{Backend Software Engineer seeking Summer 2021 Co-op}
\personalinfo{%
  % Not all of these are required!
  % You can add your own with \printinfo{symbol}{detail}
  \email{galen@galenguyer.com}
  \phone{(425) 898-3145}
  \homepage{galenguyer.com/}
  \linkedin{linkedin.com/in/galen-g/}
  \github{MasterChief-John-117}
  %% You MUST add the academicons option to \documentclass, then compile with LuaLaTeX or XeLaTeX, if you want to use \orcid or other academicons commands.
%   \orcid{orcid.org/0000-0000-0000-0000}
}

%% Make the header extend all the way to the right, if you want.
\begin{fullwidth}
\marginpar{\makesidebarheader\cvsection{Education}
\cvevent{Rochester Institute of Technology}{Software Engineering B.S.}{Expected Graduation May 2024}{}

\cvsection{Skills}
\cvsubsection{Languages}
\cvtag{C\#}
\cvtag{Python}
\cvtag{PowerShell}
\cvtag{React}
\cvtag{JavaScript}
\cvtag{TypeScript}
\cvtag{Java}

\cvsubsection{Tools}
\cvtag{Azure}
\cvtag{NGINX}
\cvtag{Linux}
\cvtag{Git}
\cvtag{Bash}

%% Yeah I didn't spend too much time making all the
%% spacing consistent... sorry. Use \smallskip, \medskip,
%% \bigskip, \vpsace etc to make ajustments.
\medskip
\cvsection{Activities}

\cvevent{Computer Science House}{Member}{August 2019 - Present}{}
Maintain House Services such as virtual machine setup and account management portals and \\ manage server resources for an annual Minecraft charity stream

\divider

\cvevent{WITR Radio}{Internal Developer}{October 2019 - Present}{}
Responsible for the upkeep and development of internal and public facing services such as the public website

\divider

\cvevent{Society of Software Engineers}{Public Relations}{December 2019 - Present}{}
Coordinated company visits and tech talks throughout the semester

\divider

\cvevent{Boy Scouts}{Eagle Scout}{September 2017}{}
Achieved Boy Scouts' highest rank after earning 26 merit badges and designing, coordinating, and leading a 100+ hour service project

}
    \vspace*{-1\baselineskip}
\makecvheader
\end{fullwidth}
%% Provide the file name containing the sidebar contents as an optional parameter to \cvsection.
%% You can always just use \marginpar{...} if you do
%% not need to align the top of the contents to any
%% \cvsection title in the "main" bar.
\vspace{.65\baselineskip}

\cvsection{Experience}

\cvevent{Microsoft}{Software Engineering Intern - One Customer Voice Team}{June 2019 -- August 2019}{Redmond, WA}
\begin{itemize}
\item Improved item grouping for the internal feedback aggregation tool
\item Defined project scope and requirements within a tech spec and documented UI and backend changes in a design doc
\item Exposed all previously hidden top level fields and automatically detected field type via ElasticSearch mappings, greatly increasing both the granularity and flexibility for users
\item Implemented frontend, backend, and tests with 100\% backend code coverage
\end{itemize}
\textit{\textbf{Tools:} C\#, ASP.NET, Azure Service Fabric, ElasticSearch, AngularJS}

\divider

\cvevent{Microsoft}{Software Engineering Intern - Office Security Penetration Testing Team}{June 2018 -- August 2018}{Redmond, WA}
\begin{itemize}
\item Automated resource gathering, provisioning, and deployment of fuzzing jobs
\item Developed a PowerShell script to create a Windows VM, automatically install Office, gather debugging symbols for Office, and send these to a Microsoft \\ Security Risk Detection server to install and start a fuzzing run
\item Implemented a server to collect, deduplicate, and report new bugs
\item Worked closely with the Microsoft Security Risk Detection team to help improve their project as it was not publicly released
\end{itemize}
\textit{\textbf{Tools:} PowerShell, Microsoft Security Risk Detection, Azure, ASP.NET}

\cvsection{Projects}

\project{GenericBot}{https://github.com/MasterChief-John-117/GenericBot}
\begin{itemize}
\item Provides moderation tools and fun commands to over 30 servers totaling over 12,000 users
\item Uses C\# to connect to Discord's API and stores all user data encrypted in a self-hosted MongoDB instance
\item Exposes user-stored quotes via a webpage with an ASP.NET API and Vue frontend for responsive search
\end{itemize}
\textit{\textbf{Tools:} C\#, MongoDB, ASP.NET, HTML, Vue, Azure Build Pipelines}

\divider

\project{PhotoSink}{https://github.com/MasterChief-John-117/PhotoSink}
\begin{itemize}
\item A re-write of the lightweight photo gallery PhotoFloat, using modern technologies
\item Uses a python3 backend to construct the directory indexes and resize images into thumbnails for faster page loads
\item The frontend is a React app that looks at the indexes to build the gallery with thumbnails and folder names
\item Makes use of the url fragment for single page navigation so the full page doesn't need to be reloaded on navigation
\end{itemize}
\textit{\textbf{Tools:} Python, React, NGINX}

%\project{QuickLinks}{https://github.com/MasterChief-John-117/QuickLinks}
%\begin{itemize}
%\item Takes long, hard to remember links and provides a three word memorable link
%\item Written in ASP.NET and uses LiteDb for a simple single-file database
%\end{itemize}
%\textit{\textbf{Tools:} ASP.NET, LiteDb}

%\divider

%\noghproject{Worldwide CDN}{https://gallery.mastrchef.rocks}
%\begin{itemize}
%\item Consists of four Azure Virtual Machines distributed around the globe to provide the fastest server to the user based on geographical location
%\item Uses inotifywait and rsync to distribute updated files from central server to all edge servers automatically
%\item Serves an attractive, lightweight photo gallery with nginx and PhotoFloat
%\end{itemize}
%\textit{\textbf{Tools:} Azure Virtual Machines, Azure Traffic Managers, Bash, Nginx}

\clearpage

\end{document}
